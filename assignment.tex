\documentclass[
            a4paper
            ]{scrartcl}%article

%\usepackage[biber]{assignment}
%\addbibresource{references.bib}
\usepackage{assignment}


%%%%%%%%%%%%%%%%%% MATH-CONSTANTS %%%%%%%%%%%%%%%%%%%%%%
\newcommand{\printCostFactor}{\pgfmathparse{\costFactor}\pgfmathprintnumber[fixed,
precision=1]{\pgfmathresult}}
\newcommand{\printpercent}[1]{\pgfmathparse{100*#1}\pgfmathprintnumber[fixed,
precision=3]{\pgfmathresult}\%}
\pgfmathsetmacro{\alphas}{0.96899}      %Availability value simple
\pgfmathsetmacro{\alphar}{0.99887}      %Availability value redundant
\pgfmathsetmacro{\workingCosts}{10}     %working-costs of simple System
\pgfmathsetmacro{\costFactor}{2.5}      %factor, the redundant-system is more
                                        %   expensive, than simple system

%%%%%%%%%%%%%%%%%% MATH-FUNCTIONS %%%%%%%%%%%%%%%%%%%%%%
\pgfmathdeclarefunction{kst1}{3}{       %kst1(Omega, alpha_s, k)
	\pgfmathparse{(#1*#2)+((1-#2)*#1*#3)}
}
\pgfmathdeclarefunction{kst2}{3}{ %kst2(Omega, alpha_r, k)
	\pgfmathparse{(#1*\costFactor*#2)+((1-#2)*#1*#3)}
}
\pgfmathdeclarefunction{schnittpunkt}{2} { %schnittpunkt(alphar, alphas)
	\pgfmathparse{(\costFactor*#1-#2)/(#1-#2)}
}
\pgfmathsetmacro{\punktkx}{schnittpunkt(\alphar, \alphas)}

\title{Praktisches Übungsbeispiel SS2014}
\subtitle{Zuverlässigkeitsmodellierung}
\rohead{Übungsbeispiel SS14}
\subject{VU Dependable Systems 182.712}

\author{
 \authorname{Markus Kessler} \\
 \studentnumber{1225380} \\
 \curriculum{033 535}\\
 \email{e1225380@student.tuwien.ac.at}\\\\
 \authorname{Mathias Lechner} \\
 \studentnumber{1225134} \\
 \curriculum{033 535}\\
 \email{e1225134@student.tuwien.ac.at}\\\\
 \authorname{Martin Wührer} \\
 \studentnumber{1225177} \\
 \curriculum{033 535}\\
 \email{e1225177@student.tuwien.ac.at}
}

\lohead{Kessler, Lechner, Wührer}

\begin{document}

\renewcommand*{\Frefeqname}{Gleichung}
\renewcommand*{\Freffigname}{Abbildung}
\renewcommand*{\Frefsecname}{Abschnitt}

\maketitle
\newpage

\tableofcontents
\newpage
\section{Abstract}
\begin{abstract}
Our task is to create a Markov chain model for simulating the failure of a
computer system. The MTTF\footnote{MTTF\dots Mean Time to Failure} and the 
availability of a simple and a redundant version of the system are compared. 
Additionally a cost evaluation between the two systems is considered.
The computation of the results is done with \emph{SHARPE}.
\end{abstract}

\section{Executive Summary}
We created Markov models to compute the MTTF for the following systems:
	\begin{enumerate}[a)]
		\item simple
		\item redundant
		\item redundant with repair variant of the system
	\end{enumerate}
	The failure probability of the three networks were calculated and plotted. The limit of the proportion between the operative and the failure costs was computed with the help of the availability values of the systems. We are going to demonstrate the improvements and benefits from maintenance and redundancy.
\newpage

\section{Problemstellung}
\paragraph{Einfache Variante:}
Ein Embedded Computersystem besteht aus drei über switched Ethernet verbundene
Knoten (siehe \Fref{fig:simple_variant}).

Die Fehlerrate jedes einzelnen Knoten beträgt $\lambda_R=10^{-4}/Std.$ und die Fehlerrate
des Netzwerks beträgt $\lambda_N=2\cdot 10^{-5}$.
Die Reparaturraten von Netzwerk und Node sind identisch und betragen $\mu_R=\mu_N=10^{-2}/Std.$

Im Falle eines Node- bzw. Netzwerkausfalls kann das System jedoch nicht mehr weiterarbeiten.

\paragraph{Redundante Erweiterung:}
Um diesem Problem vorzubeugen und die Funktionsfähigkeit zu verbessern, wir das System um einen Node sowie einen Switch erweitert (siehe \Fref{fig:redundant_variant}). Der zusätzliche Knoten übernimmt im Falle eines Nodeausfalls sofort dessen Arbeit. Das gleiche gilt für den Switch, es entsteht somit keine Down-Time, das System bleibt \emph{fail-operational}.

Auch für diese redundante Variante gilt, dass mindestens 3 Knoten und ein Netzwerk funktionieren
müssen. Es gelten dabei die Fehler- und Reparaturraten der einfachen Variante.

Der Betrieb dieser erweiterten Variante ist jedoch mit zusätzlichen
Anschaffungskosten und laufenden Betriebskosten etwa \printCostFactor-mal so
teuer, wie der Normalbetrieb der einfachen Variante.

\paragraph{Aufgaben:} Die Evaluierung, welches System für den Betrieb besser geeignet ist,
erfordert einen genauen Vergleich der beiden Systeme. Dazu sind folgende 2
Schritte nötig:
\begin{enumerate}[i)]
    \item Berechnen und vergleichen der MTTF des einfachen und des redundanten Systems. Wobei beim
        redundanten System die MTTF ohne und mit Reparatur berechnet werden
        soll. (für die Berechnung siehe \Fref{sec:sol_mttf})
    \item Wenn das System nicht läuft, verursacht es Ausfallskosten, die ein
        Vielfaches der Betriebskosten der einfachen Variante betragen. 
        Es soll somit das Verhältnis von laufenden Kosten zu den Ausfallskosten
        berechnet werden, ab dem die redundante Variante der einfachen Variante
        vorzuziehen ist (für die Berechnung siehe \Fref{sec:sol_avail}).
\end{enumerate}
\begin{figure}
    \centering
    \begin{subfigure}[b]{0.48\linewidth}
    \centering
    \scalebox{.85}{
    \begin{tikzpicture}
        \tikzstyle{node} = [top color=white, bottom color=blue!30, 
                                draw=blue!50!black!100, drop shadow]
        \tikzstyle{switch} = [top color=white, bottom color=red!30, 
                                draw=red!50!black!100, drop shadow,
                                rounded corners=5pt]
        \node[switch](sw1){Switch};
        \node[node] (nd1)   [above left=of sw1]{Node 1};
        \node[node] (nd2)   [above=of sw1]{Node 2};
        \node[node] (nd3)   [above right=of sw1]{Node 3};

        \path (sw1) edge (nd1);
        \path (sw1) edge (nd2);
        \path (sw1) edge (nd3);
    \end{tikzpicture}}
    \caption{Einfache Variante}
    \label{fig:simple_variant}
    \end{subfigure}
    \begin{subfigure}[b]{0.48\linewidth}
    \scalebox{.85}{
    \begin{tikzpicture}
        \tikzstyle{node} = [top color=white, bottom color=blue!30, 
                                draw=blue!50!black!100, drop shadow]
        \tikzstyle{backup} = [bottom color=green!30]
        \tikzstyle{switch} = [top color=white, bottom color=red!30, 
                                draw=red!50!black!100, drop shadow,
                                rounded corners=5pt]

        \node[switch](sw1){Switch 1};
        \node[switch,backup](sw2)  [right=of sw1]{Switch 2};
        \node[node] (nd1)   [above left=of sw1]{Node 1};
        \node[node] (nd2)   [above=of sw1]{Node 2};
        \node[node] (nd3)   [above=of sw2]{Node 3};
        \node[node,backup] (nd4)   [above right=of sw2]{Node 4};

        \path (sw1) edge (nd1);
        \path (sw1) edge (nd2);
        \path (sw1) edge (nd3);
        \path (sw1) edge (nd4);

        \path (sw2) edge (nd1);
        \path (sw2) edge (nd2);
        \path (sw2) edge (nd3);
        \path (sw2) edge (nd4);
    \end{tikzpicture}}
    \caption{Fehlertolerant erweiterte Variante}
    \label{fig:redundant_variant}
    \end{subfigure}
    \caption{Einfache und fehlertolerant erweiterte Variante des Computersystems}
\end{figure}

\newpage
\section{Lösungsmethode}
\subsection{MTTF}\label{sec:sol_mttf}

\paragraph{Einfache Variante:}

\begin{figure}
    \centering
    \begin{subfigure}[b]{0.48\linewidth}
        \centering
        \begin{tikzpicture}[>=latex]
            \tikzstyle{markov} = [top color=blue!30, bottom color=white, 
                            draw=blue!50!black!100, drop shadow]
            \tikzstyle{bad} = [top color=red!30]

            \node[markov,circle split,label=below left:{\color{green!70!black}\textbf{a}}] (1) {$3/3$ \nodepart{lower} $1/1$};
            \node[markov,bad,circle split,label=below right:{\color{green!70!black}\textbf{b}}] (2) [above right=of 1]{$2/3$ \nodepart{lower} $1/1$};
            \node[markov,bad,circle split,label=below right:{\color{green!70!black}\textbf{c}}] (3) [below right=of 1]{$3/3$ \nodepart{lower} $0/1$};
            \draw[->] (1) to node[sloped,above]{$3\lambda_R$}(2);
            \draw[->] (1) to node[sloped,above]{$\lambda_N$}(3);
        \end{tikzpicture}
        \caption{\textbf{Zustand a}: Funktionierend, \\
            \textbf{Zustand b}: Rechnerausfall, \\
            \textbf{Zustand c}: Netzwerkausfall}
        \label{fig:states_simple_mttf}
    \end{subfigure}
    \begin{subfigure}[b]{0.48\linewidth}
        \centering
            \begin{tikzpicture}[>=latex]
                \tikzstyle{markov} = [top color=blue!30, bottom color=white, 
                                draw=blue!50!black!100, drop shadow]
                \tikzstyle{bad} = [top color=red!30]

                \node[markov,circle,label=below right:{\color{green!70!black}\textbf{a}}] (1) {alive};
                \node[markov,bad,circle,label=below right:{\color{green!70!black}\textbf{b}}] (2)
                [right=3 of 1]{dead};
                \draw[->] (1) to node[sloped,above]{$3\lambda_R + \lambda_N$}(2);
            \end{tikzpicture}
            \caption{Vereinfachte Variante von \Fref{fig:states_simple_mttf} \\
                \textbf{Zustand a}: System funktioniert, \\
                \textbf{Zustand b}: System funktioniert nicht}
            \label{fig:states_simple_mttf_simply}
    \end{subfigure}
    \caption{Möglichen Zustände der einfachen Variante.\\
        Die Zustände in denen das System lauffähig ist, sind hier Blau dargestellt.}
    \label{fig:states_simple_mttf_comb}
\end{figure}

Dieses Modell besitzt die folgenden Zustände: (siehe \Fref{fig:states_simple_mttf})
\begin{enumerate}[\bfseries a.]
    \item das System arbeitet fehlerfrei ($3/3$ | $1/1$)
    \item ein Node ist defekt ($2/3$ | $1/1$)
    \item ein Switch ist defekt ($3/3$ | $0/1$)
\end{enumerate}

Da es allerdings keine Rolle spielt, ob ein Rechner oder das Netzwerk ausfällt
(in beiden Fällen ist das System nicht mehr lauffähig), kann das
Modell noch stärker vereinfacht werden. 
Im Folgenden wird nur noch von den Zuständen \enquote{alive} und \enquote{dead} gesprochen 
(siehe \Fref{fig:states_simple_mttf_simply}).

Bei diesem Modell werden die Übergangswahrscheinlichkeiten vom laufenden in die
defekten Zustände einfach addiert.

\paragraph{Redundante Variante ohne Reparatur:}
\begin{figure}
    \centering
        \begin{tikzpicture}[>=latex]
            \tikzstyle{markov} = [top color=blue!30, bottom color=white, 
                            draw=blue!50!black!100, drop shadow]
            \tikzstyle{bad} = [top color=red!30]

            \node[markov,circle split,label=below left:{\color{green!70!black}\textbf{a}}] (42) {$4/4$ \nodepart{lower} $2/2$};
            \node[markov,circle split,label=above right:{\color{green!70!black}\textbf{b}}] (32) [above right=of 42]{$3/4$ \nodepart{lower} $2/2$};
            \node[markov,circle split,label=below right:{\color{green!70!black}\textbf{d}}] (31) [below right=of 32]{$3/4$ \nodepart{lower} $1/2$};
            \node[markov,circle split,label=below right:{\color{green!70!black}\textbf{c}}] (41) [below right=of 42]{$4/4$ \nodepart{lower} $1/2$};
            \node[markov,bad,circle,label=below right:{\color{green!70!black}\textbf{e}}] (d)  [right=3 of 31]{dead};
            \draw[->] (42) to node[sloped,above]{$4\lambda_R$}(32);
            \draw[->] (42) to node[sloped,above]{$2\lambda_N$}(41);
            \draw[->] (41) to node[sloped,above]{$4\lambda_R$}(31);
            \draw[->] (32) to node[sloped,above]{$2\lambda_N$}(31);
            \draw[->] (32) to[bend left=20] node[sloped,above]{$3\lambda_R$}(d);
            \draw[->] (31) to node[sloped,above]{$3\lambda_R+\lambda_N$}(d);
            \draw[->] (41) to[bend right=20] node[sloped,above]{$\lambda_N$}(d);
        \end{tikzpicture}
    \caption{Mögliche Zustände der redundanten Variante ohne Reparatur.\\
        Die Zustände in denen das System lauffähig ist, sind hier Blau dargestellt.}
    \label{fig:states_redundant_without_rep}
\end{figure}

Dieses Modell besitzt bereits viel mehr Zustände, da hier auch mit ausgefallenen
Komponenten noch ein laufender Zustand möglich ist. (siehe
\Fref{fig:states_redundant_without_rep})
\begin{enumerate}[\bfseries a.]
    \item das System arbeitet fehlerfrei ($4/4$ | $2/2$)
    \item ein Node ist defekt ($3/4$ | $2/2$)
    \item ein Swich ist defekt ($4/4$ | $1/2$)
    \item ein Node sowie ein Switch sind defekt ($3/4$ | $1/1$)
    \item das System ist nicht mehr lauffähig (\enquote{dead})
\end{enumerate}
Jedoch ist diese Variante für die Praxis eher ungeeignet, denn ein redundantes
System ohne Reparatur macht meist wenig Sinn. Besser ist hier die
Variante mit Reparatur (siehe Lösung in \Fref{sec:results_mttf}).

\paragraph{Redundante Variante mit Reparatur:}
\begin{figure}
    \centering
        \begin{tikzpicture}[>=latex]
            \tikzstyle{markov} = [top color=blue!30, bottom color=white, 
                            draw=blue!50!black!100, drop shadow]
            \tikzstyle{bad} = [top color=red!30]

            \node[markov,circle split,label=below left:{\color{green!70!black}\textbf{a}}] (42) {$4/4$ \nodepart{lower} $2/2$};
            \node[markov,circle split,label=above right:{\color{green!70!black}\textbf{b}}] (32) [above right=of 42]{$3/4$ \nodepart{lower} $2/2$};
            \node[markov,circle split,label=below right:{\color{green!70!black}\textbf{d}}] (31) [below right=of 32]{$3/4$ \nodepart{lower} $1/2$};
            \node[markov,circle split,label=below right:{\color{green!70!black}\textbf{c}}] (41) [below right=of 42]{$4/4$ \nodepart{lower} $1/2$};
            \node[markov,bad,circle,label=below right:{\color{green!70!black}\textbf{e}}] (d)  [right=3 of 31]{dead};
            \draw[->] (42) to node[sloped,above]{$4\lambda_R$}(32);
            \draw[->] (32) to[bend right=45] node[sloped,above]{$\mu_R$}(42);
            \draw[->] (42) to node[sloped,above]{$2\lambda_N$}(41);
            \draw[->] (41) to[bend right=-45] node[sloped,above]{$\mu_N$}(42);
            \draw[->] (41) to node[sloped,above]{$4\lambda_R$}(31);
            \draw[->] (31) to node[sloped,above]{$\mu_N$} (42);
            \draw[->] (32) to node[sloped,above]{$2\lambda_N$}(31);
            \draw[->] (32) to[bend left=20] node[sloped,above]{$3\lambda_R$}(d);
            \draw[->] (31) to node[sloped,above]{$3\lambda_R+\lambda_N$}(d);
            \draw[->] (41) to[bend right=20] node[sloped,above]{$\lambda_N$}(d);
        \end{tikzpicture}
    \caption{Mögliche Zustände der redundanten Variante mit Reparatur.\\
        Die Zustände in denen das System lauffähig ist, sind hier Blau dargestellt.}
    \label{fig:states_redundant_with_rep}
\end{figure}

Dieses Modell besitzt die gleichen Zustände, wie das System ohne Reparatur, jedoch
sind die zusätzlichen Zustandswechsel nach der Reparatur auch noch
eingezeichnet (siehe \Fref{fig:states_redundant_with_rep}). 
\begin{enumerate}[\bfseries a.]
    \item das System arbeitet fehlerfrei ($4/4$ | $2/2$)
    \item ein Node ist defekt ($3/4$ | $2/2$)
    \item ein Swich ist defekt ($4/4$ | $1/2$)
    \item ein Node sowie ein Switch sind defekt ($3/4$ | $1/1$)
    \item das System ist nicht mehr lauffähig (\enquote{dead})
\end{enumerate}

Für den Fall (\textbf{d}) wird davon ausgegangen, dass der Techniker beide Fehler 
gleichzeitig beheben kann, was in der Praxis nicht unrealistisch ist.
So werden beispielsweise in großen Rechenzentren die Techniker immer mit genügend
Ersatzkomponenten ausgestattet, um diese möglichst rasch durch die defekten Komponenten
ersetzen zu können (somit ist nur ein einfaches Ab- und Anstecken notwendig).
Erst nach dem erfolgreichen Tausch (meist in der Werkstatt) werden die
Komponenten repariert. 

Das Zeitaufwendigste an einer Reparatur ist meist, dass ein
Techniker (der oftmals nur in Bereitschaft ist) vor Ort kommen muss, um den
Komponententausch durchzuführen.

Beispielsweise kann dieser immer mit Ersatznode und -switch ausgestattet sein
und ggf. beide austauschen. 
Deshalb existiert in diesem Modell keine Möglichkeit, um von (\textbf{d}) auf
(\textbf{c}) oder (\textbf{b}) zu kommen.

\subsection{Availability}\label{sec:sol_avail}
\paragraph{Einfache Variante:}
\begin{figure}
    \centering
    \begin{subfigure}[b]{0.48\linewidth}
        \centering
        \begin{tikzpicture}[>=latex]
            \tikzstyle{markov} = [top color=blue!30, bottom color=white, 
                            draw=blue!50!black!100, drop shadow]
            \tikzstyle{bad} = [top color=red!30]

            \node[markov,circle split,label=below left:{\color{green!70!black}\textbf{a}}] (1) {$3/3$ \nodepart{lower} $1/1$};
            \node[markov,bad,circle split,label=below right:{\color{green!70!black}\textbf{b}}] (2) [above right=of 1]{$2/3$ \nodepart{lower} $1/1$};
            \node[markov,bad,circle split,label=below right:{\color{green!70!black}\textbf{c}}] (3) [below right=of 1]{$3/3$ \nodepart{lower} $0/1$};
            \draw[->] (1) to node[sloped,above]{$3\lambda_R$}(2);
            \draw[->] (1) to node[sloped,above]{$\lambda_N$}(3);
            \draw[->, blue!70!black] (3) to[bend right=45] node[sloped,above]{$\mu_N$}(1);
            \draw[->, blue!70!black] (2) to[bend right=45] node[sloped,above]{$\mu_R$}(1);
        \end{tikzpicture}
        \caption{\textbf{Zustand a}: Funktionierend, \\
            \textbf{Zustand b}: Rechnerausfall, \\
            \textbf{Zustand c}: Netzwerkausfall}
        \label{fig:states_simple_avail}
    \end{subfigure}
    \begin{subfigure}[b]{0.48\linewidth}
        \centering
            \begin{tikzpicture}[>=latex]
                \tikzstyle{markov} = [top color=blue!30, bottom color=white, 
                                draw=blue!50!black!100, drop shadow]
                \tikzstyle{bad} = [top color=red!30]

                \node[markov,circle,label=below right:{\color{green!70!black}\textbf{a}}] (1) {alive};
                \node[markov,bad,circle,label=below right:{\color{green!70!black}\textbf{b}}] (2)
                [right=5 of 1]{dead};
                \draw[->] (1) to node[sloped,above]{$3\lambda_R + \lambda_N$}(2);
                \draw[->, blue!70!black] (2) to[bend right=45] node[sloped,above]{$\mu_R=\mu_N$}(1);
            \end{tikzpicture}
            \caption{Vereinfachte Variante von \Fref{fig:states_simple_avail} \\
                \textbf{Zustand a}: Funktionierend, \\
                \textbf{Zustand b}: nicht Funktionierend}
            \label{fig:states_simple_avail_simply}
    \end{subfigure}
    \caption{Zustände der einfachen Variante zur Berechnung der Availability.\\
        Die Zustände in denen das System lauffähig ist, sind hier Blau
    dargestellt.}
    \label{fig:states_simple_avail_comb}
\end{figure}
Zur Berechnung der Availability besitzt dieses Modell die selben Zustände, die
auch zur Berechnung der MTTF verwendet wurden. Jedoch wird zusätzlich ein
Übergang vom Zustand \enquote{dead} zu dem Zustand \enquote{alive} eingeführt.
Auch hier wird angenommen, dass der Techniker alle Fehler gleichzeitig beheben
kann und somit direkt in den Zustand \enquote{alive} (alle Komponenten
funktionieren) wechseln kann (siehe \Fref{fig:states_simple_avail_comb}).

\paragraph{Redundante Variante:}
\begin{figure}
    \centering
        \begin{tikzpicture}[>=latex]
            \tikzstyle{markov} = [top color=blue!30, bottom color=white, 
                            draw=blue!50!black!100, drop shadow]
            \tikzstyle{bad} = [top color=red!30]

            \node[markov,circle split,label=below left:{\color{green!70!black}\textbf{a}}] (42) {$4/4$ \nodepart{lower} $2/2$};
            \node[markov,circle split,label=above right:{\color{green!70!black}\textbf{b}}] (32) [above right=of 42]{$3/4$ \nodepart{lower} $2/2$};
            \node[markov,circle split,label=below right:{\color{green!70!black}\textbf{d}}] (31) [below right=of 32]{$3/4$ \nodepart{lower} $1/2$};
            \node[markov,circle split,label=below right:{\color{green!70!black}\textbf{c}}] (41) [below right=of 42]{$4/4$ \nodepart{lower} $1/2$};
            \node[markov,bad,circle,label=below right:{\color{green!70!black}\textbf{e}}] (d)  [right=3 of 31]{dead};
            \draw[->] (42) to node[sloped,above]{$4\lambda_R$}(32);
            \draw[->] (32) to[bend right=45] node[sloped,above]{$\mu_R$}(42);
            \draw[->] (42) to node[sloped,above]{$2\lambda_N$}(41);
            \draw[->] (41) to[bend right=-45] node[sloped,above]{$\mu_N$}(42);
            \draw[->] (41) to node[sloped,above]{$4\lambda_R$}(31);
            \draw[->] (31) to node[sloped,above]{$\mu_N$} (42);
            \draw[->] (32) to node[sloped,above]{$2\lambda_N$}(31);
            \draw[->] (32) to[bend left=20] node[sloped,above]{$3\lambda_R$}(d);
            \draw[->] (31) to node[sloped,above]{$3\lambda_R+\lambda_N$}(d);
            \draw[->] (41) to[bend right=20] node[sloped,above]{$\lambda_N$}(d);
            \draw[->,blue!70!black] (d) to[bend right=108] node[sloped,above]{$\mu_N=\mu_R$}(42);
        \end{tikzpicture}
    \caption{Zustände der redundanten Variante zur Berechnung der Availability.\\
        Die Zustände in denen das System lauffähig ist, sind hier Blau
    dargestellt.}
    \label{fig:states_redundant_avail}
\end{figure}
Zur Berechnung der Availability von diesem System wird hier ebenso ein Übergang von \enquote{dead} 
zum Startzustand eingeführt. 
Für die Berechnung der Verfügbarkeit wird definiert, dass das System noch
lauffähig ist, wenn es sich in einem der blauen Zustände befindet (siehe
\Fref{fig:states_redundant_avail}).

\paragraph{Kostenvergleich:}
Um nun die Kosten der beiden Varianten zu vergleichen, muss von beiden Systemen
die Availability berechnet werden (\Fref{sec:res_avail}). Erst dann können die Ausfallskosten zu den
laufenden Kosten in Relation gestellt und verglichen werden (siehe \Fref{sec:cost_calc}).

\section{Ergebnisse}

\subsection{MTTF}\label{sec:results_mttf}
\begin{figure}
\centering
\begin{tikzpicture}
    \begin{axis}[
        height=5.2cm, width=12cm,
        ymin=0, ymax=1,
        axis lines=center,
        axis on top=true,
        domain=0:1,
        legend pos=south east,
        %legend style={at={(axis cs:1800,0.2)},anchor=south east},
        xlabel=Tage,
        ylabel=Ausfallswahrscheinlichkeit,
        x label style={at={(axis description cs:1.07,0.07)},anchor=north},
        y label style={at={(axis description cs:-0.07,.5)},rotate=90,anchor=south},
        %ylabel=Number of failed banks
        ] % use TeX as calculator: 
    \addplot[blue,thick,domain=0:1800,samples=100] table {simple_mttf.dat}; 
    \addlegendentry{Simple Variante}
    \addplot[green,thick,domain=0:1800,samples=100] table {redundant_without_repair_mttf.dat}; 
    \addlegendentry{Red. Variante (o. Rep.)}
    \addplot[red,thick,domain=0:1800,samples=100] table {redundant_mttf.dat}; 
    \addlegendentry{Red. Variante (m. Rep.)}
    \end{axis} 
\end{tikzpicture}

\caption{Vergleich der MTTF der simplen Variante (siehe
    \Fref{fig:states_simple_mttf_simply}), \\
    der redundanten Variante ohne Reparatur (siehe
    \Fref{fig:states_redundant_without_rep})\\
    und der redundanten Variante mit Reparatur (siehe \Fref{fig:states_redundant_with_rep})}
\label{fig:mttf_result}
\end{figure}

\begin{table}
\centering
\pgfkeys{/pgf/fpu=true}
\pgfkeys{/pgf/number format,use comma,fixed,precision=4}

\pgfplotstableset{
    every head row/.style={before row=\hline, after row=\hline},
    every last row/.style={after row=\hline},
    create on use/tmp/.style={create col/set list={1,2,3,...,50}}}
\pgfplotstableread{simple_mttf.dat}\loadedtableSimple
\pgfplotstableread{redundant_mttf.dat}\loadedtableRed
\pgfplotstableread{redundant_without_repair_mttf.dat}\loadedtableRedWithout

\pgfplotstablenew[columns=tmp]{50}\tbl
\pgfplotstablecreatecol[copy column from table={\loadedtableSimple}{[index]0}]
{Tage}{\tbl}
\pgfplotstablecreatecol[copy column from table={\loadedtableSimple}{[index] 1}] 
{simpel}{\tbl}
\pgfplotstablecreatecol[copy column from table={\loadedtableRedWithout}{[index] 1}] 
{redund. ohne}{\tbl}
\pgfplotstablecreatecol[copy column from table={\loadedtableRed}{[index] 1}]
{redund. mit}{\tbl}

%\pgfplotstabletypeset\tbl

%\pgfplotstabletypeset[skip first n=4]\loadedtableSimple

\pgfplotstabletypeset[
    columns={Tage,simpel, redund. ohne, redund. mit,Tage,simpel, redund. ohne,
        redund. mit},
    display columns/0/.style={select equal part entry of={0}{2}},
    display columns/1/.style={select equal part entry of={0}{2}},
    display columns/2/.style={select equal part entry of={0}{2}},
    display columns/3/.style={
        select equal part entry of={0}{2}, column type/.add={}{|}},
    display columns/4/.style={select equal part entry of={1}{2}},
    display columns/5/.style={select equal part entry of={1}{2}},
    display columns/6/.style={select equal part entry of={1}{2}},
    display columns/7/.style={select equal part entry of={1}{2}},
]\tbl
\pgfkeys{/pgf/fpu=false}
\caption{Mit \emph{SHARPE} ermittelte Datenwerte:\\
    Wobei hier der Spaltenname \enquote{redund. ohne} das Modell mit Redundanz,
    aber ohne Reparatur bezeichnet und der Spaltenname \enquote{redund. mit} das
    Modell mit Redundanz und mit Reparatur bezeichnet.}
\label{tab:mttf_result}
\end{table}




Die Ergebnisse der MTTF Simulation sind in \Fref{fig:mttf_result} dargestellt.
Diese Ergebnisse wurden mit Hilfe der \emph{SHARPE}-Programme \Fref{sec:calc_mttf_simple},
\Fref{sec:calc_mttf_redundant_rep} und \Fref{sec:calc_mttf_redundant} berechnet.
Für die genauen Werte, siehe \Fref{tab:mttf_result}.

\subsection{Availability}\label{sec:res_avail}
Die Availability $\alpha_s$ des einfachen Systems lässt sich als Zustandswahrscheinlichkeit der Markov Kette (siehe \Fref{fig:states_simple_avail_simply}) des Zustands \textbf{b} berechnen. Dies kann mit dem \emph{SHARPE} Befehl \mbox{\texttt{expr prob(SYS,1)}} bewerkstelligt werden. \\
Mit den gegebenen Werten ergibt sich: $\alpha_s = \printpercent{\alphas}$ \\

Die Availability $\alpha_r$ des fehlertoleranten Systems entspricht der Wahrscheinlichkeit, sich in einem der Zustände $\lbrace a, b, c, d \rbrace$ zu befinden (siehe \Fref{fig:states_redundant_avail}). Dies ist äquivalent mit der Zustandswahrscheinlichkeit nicht im Zustand \textbf{e} (System Ausfall) zu sein und kann mittels \mbox{\texttt{expr 1-prob(SYS,5)}} berechnet werden (Zustand 5 = Zustand e).\\
In unserem Fall ergibt das: $\alpha_r = \printpercent{\alphar}$\\

Die \emph{SHARPE}-Programme für die Berechnung von $\alpha_s$ und $\alpha_r$ siehe \Fref{sec:calc_aval_simple} und \Fref{sec:calc_aval_redundant}.\\

\section{Diskussion}
\subsection{MTTF}
Beim Vergleich der MTTF Graphen (siehe \Fref{fig:mttf_result}), wurde festgestellt, dass sich die 
Ausfallwahrscheinlichkeit der redundanten Variante mit Reparatur deutlich von den beiden anderen 
Systemen abhebt.

Das bedeutet, eine redundante Variante ohne Reparatur ist in etwa vergleichbar
mit der simplen Variante. Nur dass die redundante Variante natürlich
in der Anschaffung und im Betrieb teurer ist.

Während die einfache Variante bereits nach einem Jahr mit einer
Wahrscheinlichkeit von über 98\% ausgefallen ist, funktioniert das redundante
System nach der gleichen Zeit noch mit einer Wahrscheinlichkeit von 20\%. Dieser
Wert reduziert sich allerdings im zweiten Jahr auf unter 2\%. \\
Die redundante Variante mit Reparatur, ist selbst nach 8 Jahren (2920 Tagen) noch mit
über 45\%-iger Wahrscheinlichkeit im Betrieb. 
Erst nach über 16 Jahren (etwa bei 5950 Tagen) steigt die
Ausfallwahrscheinlichkeit auch hier über 80\%.
\subsection{Kostenrechnung}\label{sec:cost_calc}
Aus der Simulation mit \emph{SHARPE} können die beiden Availability Werte
$\alpha_s$ für das einfache System und $\alpha_r$ für das redundante System
ermittelt werden.\\
Sei $\Omega$ die Betriebskosten des einfachen Systems, so lassen sich die Kosten beschreiben mit:
\begin{equation}
C_s(k) = \Omega \cdot \alpha_s + (1-\alpha_s)\cdot \Omega \cdot k
\end{equation}

wobei $k \cdot \Omega$ die Ausfallkosten des Systems sind ($k \in
\mathbb{N}^+$).\\$k$ stellt das Verhältnis zwischen Laufender Kosten und Ausfallkosten des einfachen Systems dar.\\
Die Kosten des redundanten Systems ergeben sich mit: 
\begin{equation}
C_r(k) = \Omega \cdot \printCostFactor \alpha_r + (1- \alpha_r) \cdot \Omega \cdot k
\end{equation}

Um das Verhältnis $k$ zu bestimmen, ab dem sich der Betrieb des redundanten Systems auszahlt, muss der Schnittpunkt $k_x$ bestimmt werden:
\begin{equation}
\begin{split}
C_s(k_x) = C_r(k_x) \\ &
 \Leftrightarrow
\Omega \cdot \alpha_s + (1-\alpha_s)\Omega \cdot k_x = \Omega \cdot \printCostFactor \alpha_r + (1- \alpha_r) \cdot \Omega \cdot k_x \\ & \Leftrightarrow
k_x \cdot (1 - \alpha_s - 1 + \alpha_r) = \printCostFactor \alpha_r - \alpha_s \\ & \Leftrightarrow
k_x = \frac{\printCostFactor \alpha_r - \alpha_s}{\alpha_r - \alpha_s}
\end{split}
\end{equation}

\begin{figure}
\centering
\begin{tikzpicture}
    \begin{axis}[
	    height=5.2cm, width=12cm,
		axis lines=center,
		axis on top=true,
		domain=0:100,
    	xmin=0, xmax=100,
    	xlabel=k,
        ymin=0, ymax=50,
        ylabel=Kosten $\Omega$,
        ytick = {0,10,...,50},
        x label style={at={(axis description cs:1.05,0.05)},anchor=north},
        y label style={at={(axis description cs:-0.07,.5)},rotate=90,anchor=south},
        legend style={at={(axis cs:100, 2)},anchor=south east},
        %ylabel=Number of failed banks
        ] % use TeX as calculator: 
    \addplot +[mark=none] {kst1(\workingCosts, \alphas, x)};
    \addlegendentry{Kosten simple}
    \addplot +[mark=none] {kst2(\workingCosts, \alphar, x)};
    \addlegendentry{Kosten redundant}
    \draw[green] (axis cs:\punktkx,0) -- (axis cs:\punktkx,35) node [above]{$k_x$};
    \end{axis} 
\end{tikzpicture}
\caption{Vergleich der Kostenfunktionen mit Schnittpunkt $k_x$}
\label{fig:cost_result}
\end{figure}


%Enable fixpoint-arithmetic only for this short calculation:
\pgfkeys{/pgf/fpu=true}
\pgfkeys{/pgf/number format,use comma,fixed,precision=5}

Mit den beiden von \emph{SHARPE} berechneten Werten ergibt sich:
\[k_x = \frac{\printCostFactor \cdot \printpercent{\alphar} -
\printpercent{\alphas}}{\printpercent{\alphar} - \printpercent{\alphas}} =
\pgfmathprintnumber[fixed, precision=3]{\punktkx} \approx
\pgfmathprintnumber[fixed, precision=0]{\punktkx} \]

\pgfkeys{/pgf/fpu=false}

\section{Fazit}
Wie in \Fref{sec:results_mttf} gesehen wurde, hebt sich das redundante vom einfachen System erst mit zusätzlicher Reparatur stark von Ersterem ab. Dies ist der im Verhältnis sehr hohen \emph{repair-rate} zu verdanken.

Um das Verhältnis der Ausfalls- und Betriebskosten zu veranschaulichen kann
folgendes Zahlenbeispiel angenommen werden:\\
Für das einfache System werden im Betrieb am Tag \num{1000}€ veranschlagt und
für einen Ausfall werden pro Tag \num{50000}€ angenommen, dann zahlt sich dieses
einfache System mathematisch gesehen noch aus. \\
Es ist daher durchaus gerechtfertigt, als mögliche Variante auch über das
einfache System nachzudenken, denn erst bei Ausfallskosten über \num{51144}€
rechnet sich das redundante System.
\newpage
\appendix
\section{Sharpe-Quellcode}
\subsection{MTTF}
\subsubsection{MTTF Ohne Redundanz}\label{sec:calc_mttf_simple}
\lstinputlisting{simple_mttf.shp}

\subsubsection{MTTF mit Redundanz inklusive Reparatur}\label{sec:calc_mttf_redundant_rep}
\lstinputlisting{redundant_mttf.shp}

\subsubsection{MTTF mit Redundanz ohne Reparatur}\label{sec:calc_mttf_redundant}
\lstinputlisting{redundant_without_repair_mttf.shp}
\newpage
\subsection{Availability}
\subsubsection{Availability Ohne Redundanz}\label{sec:calc_aval_simple}
\lstinputlisting{simple_availability.shp}

\subsubsection{Availability mit Redundanz}\label{sec:calc_aval_redundant}
\lstinputlisting{redundant_availability.shp}
\end{document}
