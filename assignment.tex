\documentclass[
            a4paper
            ]{scrartcl}%article

%\usepackage[biber]{assignment}
%\addbibresource{references.bib}
\usepackage{assignment}


%%%%%%%%%%%%%%%%%% MATH-CONSTANTS %%%%%%%%%%%%%%%%%%%%%%
\newcommand{\printCostFactor}{\pgfmathparse{\costFactor}\pgfmathprintnumber[fixed,
precision=1]{\pgfmathresult}}
\newcommand{\printpercent}[1]{\pgfmathparse{100*#1}\pgfmathprintnumber[fixed,
precision=3]{\pgfmathresult}\%}
\pgfmathsetmacro{\alphas}{0.96899}      %Availability value simple
\pgfmathsetmacro{\alphar}{0.99887}      %Availability value redundant
\pgfmathsetmacro{\workingCosts}{10}     %working-costs of simple System
\pgfmathsetmacro{\costFactor}{2.5}      %factor, the redundant-system is more
                                        %   expensive, than simple system

%%%%%%%%%%%%%%%%%% MATH-FUNCTIONS %%%%%%%%%%%%%%%%%%%%%%
\pgfmathdeclarefunction{kst1}{3}{       %kst1(Omega, alpha_s, k)
	\pgfmathparse{(#1*#2)+((1-#2)*#1*#3)}
}
\pgfmathdeclarefunction{kst2}{3}{ %kst2(Omega, alpha_r, k)
	\pgfmathparse{(#1*\costFactor*#2)+((1-#2)*#1*#3)}
}
\pgfmathdeclarefunction{schnittpunkt}{2} { %schnittpunkt(alphar, alphas)
	\pgfmathparse{(\costFactor*#1-#2)/(#1-#2)}
}
\pgfmathsetmacro{\punktkx}{schnittpunkt(\alphar, \alphas)}

\title{Praktisches Übungsbeispiel SS2014}
\subtitle{Zuverlässigkeitsmodellierung}
\rohead{Übungsbeispiel SS14}
\subject{VU Dependable Systems 182.712}

\author{
 \authorname{Markus Kessler} \\
 \studentnumber{1225380} \\
 \curriculum{033 535}\\
 \email{e1225380@student.tuwien.ac.at}\\\\
 \authorname{Mathias Lechner} \\
 \studentnumber{1225134} \\
 \curriculum{033 535}\\
 \email{e1225134@student.tuwien.ac.at}\\\\
 \authorname{Martin Wührer} \\
 \studentnumber{1225177} \\
 \curriculum{033 535}\\
 \email{e1225177@student.tuwien.ac.at}
}

\lohead{Kessler, Lechner, Wührer}

\begin{document}

\renewcommand*{\Frefeqname}{Gleichung}
\renewcommand*{\Freffigname}{Abbildung}
\renewcommand*{\Frefsecname}{Abschnitt}

\maketitle
\newpage

\tableofcontents
\newpage
\begin{abstract}
Our task is to create a Markov chain model for simulating the failure of a
computer system. The mean time to failure and the availability of a simple and a
redundant version of the system are compared. Additionally a cost evaluation
between the more expensive redundant system and a simple system is considered.
The computation of the results is done with \emph{SHARPE}.
\end{abstract}

\section{Executive Summary}
TODO: English BLAH BLAH
\newpage

\section{Problemstellung}
\paragraph{Einfache Variante:}
Ein Embedded Computersystem besteht aus drei über switched Ethernet verbundene
Knoten. (siehe \Fref{fig:simple_variant})

Die Fehlerrate jedes einzelnen Knoten beträgt $\lambda_R=10^{-4}/Std.$ und die Fehlerrate
des Netzwerks beträgt $\lambda_N=2\cdot 10^{-5}$.
Die Reparaturrate von Netzwerk und Node ist gleich und beträgt
$\mu_R=\mu_N=10^{-2}/Std.$.

Ein Problem bei diesem System ist, wenn ein Knoten oder das Netzwerk ausfällt, kann das System nicht mehr
weiterarbeiten.

\paragraph{Redundante Erweiterung:}
Daher wird das System so angepasst, dass ein zusätzlicher Node und ein
zusätzliches Netzwerk aufgebaut wird, um etwas mehr Ausfallsicherheit zu
erhalten. (siehe \Fref{fig:redundant_variant})

Wobei hier der zusätzliche Knoten im Falle eines Ausfalls sofort die
Aufgabe jedes anderen Knoten übernehmen kann außerdem übernimmt das redundante
Netzwerk ebenfalls sofort die Arbeit. Somit kommt es zu keiner Down-Time, wenn
zwischen den Systemen umgeschalten wird.

Auch für dieses System gilt, dass mind. 3 Knoten und ein Netzwerk funktionieren
müssen. Es gelten dabei die Fehler- und Reparaturraten von der Einfachen
Variante.

Der Betrieb dieser erweiterten Variante ist jedoch mit zusätzlichen
Anschaffungskosten und laufenden Betriebskosten etwa \printCostFactor-mal so
teuer, wie der Normalbetrieb der einfachen Variante.

\paragraph{Aufgaben:} Die Evaluierung, welches System für den Betrieb besser geeignet ist,
erfordert einen genauen Vergleich der beiden Systeme. Dazu sind folgende 2
Schritte nötig:
\begin{enumerate}[i)]
    \item Berechnen und vergleichen der MTTF\footnote{MTTF\dots Mean Time to
        Failure} des einfachen und des redundanten Systems. Wobei beim
        Redundanten System die MTTF ohne und mit Reparatur berechnet werden
        soll. (für die Berechnung siehe \Fref{sec:sol_mttf})
    \item Wenn das System nicht läuft, verursacht es Ausfallskosten, die ein
        Vielfaches der Betriebskosten der einfachen Variante betragen. 
        Es soll somit das Verhältnis von laufenden Kosten zu den Ausfallskosten
        berechnet werden, ab dem die redundante Variante der einfachen Variante
        vorzuziehen ist. (für die Berechnung siehe \Fref{sec:sol_avail})
\end{enumerate}
\begin{figure}
    \centering
    \begin{subfigure}[b]{0.48\linewidth}
    \centering
    \scalebox{.85}{
    \begin{tikzpicture}
        \tikzstyle{node} = [top color=white, bottom color=blue!30, 
                                draw=blue!50!black!100, drop shadow]
        \tikzstyle{switch} = [top color=white, bottom color=red!30, 
                                draw=red!50!black!100, drop shadow,
                                rounded corners=5pt]
        \node[switch](sw1){Switch};
        \node[node] (nd1)   [above left=of sw1]{Node 1};
        \node[node] (nd2)   [above=of sw1]{Node 2};
        \node[node] (nd3)   [above right=of sw1]{Node 3};

        \path (sw1) edge (nd1);
        \path (sw1) edge (nd2);
        \path (sw1) edge (nd3);
    \end{tikzpicture}}
    \caption{Einfache Variante}
    \label{fig:simple_variant}
    \end{subfigure}
    \begin{subfigure}[b]{0.48\linewidth}
    \scalebox{.85}{
    \begin{tikzpicture}
        \tikzstyle{node} = [top color=white, bottom color=blue!30, 
                                draw=blue!50!black!100, drop shadow]
        \tikzstyle{backup} = [bottom color=green!30]
        \tikzstyle{switch} = [top color=white, bottom color=red!30, 
                                draw=red!50!black!100, drop shadow,
                                rounded corners=5pt]

        \node[switch](sw1){Switch 1};
        \node[switch,backup](sw2)  [right=of sw1]{Switch 2};
        \node[node] (nd1)   [above left=of sw1]{Node 1};
        \node[node] (nd2)   [above=of sw1]{Node 2};
        \node[node] (nd3)   [above=of sw2]{Node 3};
        \node[node,backup] (nd4)   [above right=of sw2]{Node 4};

        \path (sw1) edge (nd1);
        \path (sw1) edge (nd2);
        \path (sw1) edge (nd3);
        \path (sw1) edge (nd4);

        \path (sw2) edge (nd1);
        \path (sw2) edge (nd2);
        \path (sw2) edge (nd3);
        \path (sw2) edge (nd4);
    \end{tikzpicture}}
    \caption{Fehlertolerant erweiterte Variante}
    \label{fig:redundant_variant}
    \end{subfigure}
    \caption{Einfache und fehlertolerant erweiterte Variante des Computersystems}
\end{figure}


\section{Lösungsmethode}
\subsection{MTTF}\label{sec:sol_mttf}

\paragraph{Einfache Variante:}

\begin{figure}
    \centering
    \begin{subfigure}[b]{0.48\linewidth}
        \centering
        \begin{tikzpicture}[>=latex]
            \tikzstyle{markov} = [top color=blue!30, bottom color=white, 
                            draw=blue!50!black!100, drop shadow]
            \tikzstyle{bad} = [top color=red!30]

            \node[markov,circle split,label=below left:{\color{green!70!black}\textbf{a}}] (1) {$3/3$ \nodepart{lower} $1/1$};
            \node[markov,bad,circle split,label=below right:{\color{green!70!black}\textbf{b}}] (2) [above right=of 1]{$2/3$ \nodepart{lower} $1/1$};
            \node[markov,bad,circle split,label=below right:{\color{green!70!black}\textbf{c}}] (3) [below right=of 1]{$3/3$ \nodepart{lower} $0/1$};
            \draw[->] (1) to node[sloped,above]{$3\lambda_R$}(2);
            \draw[->] (1) to node[sloped,above]{$\lambda_N$}(3);
        \end{tikzpicture}
        \caption{\textbf{Zustand a}: Funktionierend, \\
            \textbf{Zustand b}: Rechnerausfall, \\
            \textbf{Zustand c}: Netzwerkausfall}
        \label{fig:states_simple_mttf}
    \end{subfigure}
    \begin{subfigure}[b]{0.48\linewidth}
        \centering
            \begin{tikzpicture}[>=latex]
                \tikzstyle{markov} = [top color=blue!30, bottom color=white, 
                                draw=blue!50!black!100, drop shadow]
                \tikzstyle{bad} = [top color=red!30]

                \node[markov,circle,label=below right:{\color{green!70!black}\textbf{a}}] (1) {alive};
                \node[markov,bad,circle,label=below right:{\color{green!70!black}\textbf{b}}] (2)
                [right=3 of 1]{dead};
                \draw[->] (1) to node[sloped,above]{$3\lambda_R + \lambda_N$}(2);
            \end{tikzpicture}
            \caption{Vereinfachte Variante von \Fref{fig:states_simple_mttf} \\
                \textbf{Zustand a}: System funktioniert, \\
                \textbf{Zustand b}: System funktioniert nicht}
            \label{fig:states_simple_mttf_simply}
    \end{subfigure}
    \caption{Möglichen Zustände der einfachen Variante.\\
        Die Zustände in denen das System lauffähig ist, sind hier Blau dargestellt.}
    \label{fig:states_simple_mttf_comb}
\end{figure}

Dieses Modell besitzt die folgenden Zustände: (siehe \Fref{fig:states_simple_mttf})
\begin{enumerate}[\bfseries a.]
    \item das System arbeitet fehlerfrei ($3/3$ | $1/1$)
    \item ein Node ist defekt ($2/3$ | $1/1$)
    \item ein Switch ist defekt ($3/3$ | $0/1$)
\end{enumerate}

Da es allerdings keine Rolle spielt, ob ein Rechner oder das Netzwerk ausfällt
(in beiden Fällen ist das System nicht mehr lauffähig), kann das
Modell noch stärker vereinfacht werden. 
Im folgenden wird nur noch von den Zuständen \enquote{alive} und \enquote{dead} gesprochen. 
(siehe \Fref{fig:states_simple_mttf_simply})

Bei diesem Modell werden die Übergangswahrscheinlichkeiten vom laufenden in die
Defekten Zustände einfach addiert.

\paragraph{Redundante Variante ohne Reparatur:}
\begin{figure}
    \centering
        \begin{tikzpicture}[>=latex]
            \tikzstyle{markov} = [top color=blue!30, bottom color=white, 
                            draw=blue!50!black!100, drop shadow]
            \tikzstyle{bad} = [top color=red!30]

            \node[markov,circle split,label=below left:{\color{green!70!black}\textbf{a}}] (42) {$4/4$ \nodepart{lower} $2/2$};
            \node[markov,circle split,label=above right:{\color{green!70!black}\textbf{b}}] (32) [above right=of 42]{$3/4$ \nodepart{lower} $2/2$};
            \node[markov,circle split,label=below right:{\color{green!70!black}\textbf{d}}] (31) [below right=of 32]{$3/4$ \nodepart{lower} $1/2$};
            \node[markov,circle split,label=below right:{\color{green!70!black}\textbf{c}}] (41) [below right=of 42]{$4/4$ \nodepart{lower} $1/2$};
            \node[markov,bad,circle,label=below right:{\color{green!70!black}\textbf{e}}] (d)  [right=3 of 31]{dead};
            \draw[->] (42) to node[sloped,above]{$4\lambda_R$}(32);
            \draw[->] (42) to node[sloped,above]{$2\lambda_N$}(41);
            \draw[->] (41) to node[sloped,above]{$4\lambda_R$}(31);
            \draw[->] (32) to node[sloped,above]{$2\lambda_N$}(31);
            \draw[->] (32) to[bend left=20] node[sloped,above]{$3\lambda_R$}(d);
            \draw[->] (31) to node[sloped,above]{$3\lambda_R+\lambda_N$}(d);
            \draw[->] (41) to[bend right=20] node[sloped,above]{$\lambda_N$}(d);
        \end{tikzpicture}
    \caption{Mögliche Zustände der redundanten Variante ohne Reparatur.\\
        Die Zustände in denen das System lauffähig ist, sind hier Blau dargestellt.}
    \label{fig:states_redundant_without_rep}
\end{figure}

Dieses Modell besitzt bereits viel mehr Zustände, da hier auch mit ausgefallenen
Komponenten noch ein laufender Zustand möglich ist. (siehe
\Fref{fig:states_redundant_without_rep})
\begin{enumerate}[\bfseries a.]
    \item das System arbeitet fehlerfrei ($4/4$ | $2/2$)
    \item ein Node ist defekt ($3/4$ | $2/2$)
    \item ein Swich ist defekt ($4/4$ | $1/2$)
    \item ein Node sowie ein Switch sind defekt ($3/4$ | $1/1$)
    \item das System ist defekt (\enquote{dead})
\end{enumerate}
Jedoch ist diese Variante für die Praxis eher ungeeignet, denn ein Redundantes
System ohne Reparatur macht meist wenig Sinn. Besser ist hier die Redundante
Variante mit Reparatur. (Das zeigt auch die Lösung in \Fref{sec:results_mttf})

\paragraph{Redundante Variante mit Reparatur:}
\begin{figure}
    \centering
        \begin{tikzpicture}[>=latex]
            \tikzstyle{markov} = [top color=blue!30, bottom color=white, 
                            draw=blue!50!black!100, drop shadow]
            \tikzstyle{bad} = [top color=red!30]

            \node[markov,circle split,label=below left:{\color{green!70!black}\textbf{a}}] (42) {$4/4$ \nodepart{lower} $2/2$};
            \node[markov,circle split,label=above right:{\color{green!70!black}\textbf{b}}] (32) [above right=of 42]{$3/4$ \nodepart{lower} $2/2$};
            \node[markov,circle split,label=below right:{\color{green!70!black}\textbf{d}}] (31) [below right=of 32]{$3/4$ \nodepart{lower} $1/2$};
            \node[markov,circle split,label=below right:{\color{green!70!black}\textbf{c}}] (41) [below right=of 42]{$4/4$ \nodepart{lower} $1/2$};
            \node[markov,bad,circle,label=below right:{\color{green!70!black}\textbf{e}}] (d)  [right=3 of 31]{dead};
            \draw[->] (42) to node[sloped,above]{$4\lambda_R$}(32);
            \draw[->] (32) to[bend right=45] node[sloped,above]{$\mu_R$}(42);
            \draw[->] (42) to node[sloped,above]{$2\lambda_N$}(41);
            \draw[->] (41) to[bend right=-45] node[sloped,above]{$\mu_N$}(42);
            \draw[->] (41) to node[sloped,above]{$4\lambda_R$}(31);
            \draw[->] (31) to node[sloped,above]{$\mu_N$} (42);
            \draw[->] (32) to node[sloped,above]{$2\lambda_N$}(31);
            \draw[->] (32) to[bend left=20] node[sloped,above]{$3\lambda_R$}(d);
            \draw[->] (31) to node[sloped,above]{$3\lambda_R+\lambda_N$}(d);
            \draw[->] (41) to[bend right=20] node[sloped,above]{$\lambda_N$}(d);
        \end{tikzpicture}
    \caption{Mögliche Zustände der redundanten Variante mit Reparatur.\\
        Die Zustände in denen das System lauffähig ist, sind hier Blau dargestellt.}
    \label{fig:states_redundant_with_rep}
\end{figure}

Dieses Modell besitzt die gleichen Zustände, wie das System ohne Reparatur, jedoch
sind die zusätzlichen Zustandswechsel nach der Reparatur auch noch
eingezeichnet. (siehe
\Fref{fig:states_redundant_with_rep})
\begin{enumerate}[\bfseries a.]
    \item das System arbeitet fehlerfrei ($4/4$ | $2/2$)
    \item ein Node ist defekt ($3/4$ | $2/2$)
    \item ein Swich ist defekt ($4/4$ | $1/2$)
    \item ein Node sowie ein Switch sind defekt ($3/4$ | $1/1$)
    \item das System ist defekt (\enquote{dead})
\end{enumerate}

Für den Fall (\textbf{d}) wird davon ausgegangen, dass der Techniker beide Fehler 
gleichzeitig beheben kann, was in der Praxis nicht unrealistisch ist.
So werden beispielsweise in großen Rechenzentren die Techniker immer mit genügend
Ersatzkomponenten ausgestattet, um diese möglichst rasch durch die defekten Komponenten
ersetzen zu können (somit ist nur ein einfaches Ab- und Anstecken notwendig).
Erst nach dem erfolgreichen Tausch (meist in der Werkstatt) werden die
Komponenten repariert. 

Das Zeitaufwendigste an einer Reparatur ist meist, dass ein
Techniker (der oftmals nur in Bereitschaft ist) vor Ort kommen muss, um den
Komponententausch durchzuführen.

Beispielsweise kann dieser immer mit Ersatznode und -switch ausgestattet sein
und ggf. beide austauschen. 
Deshalb existiert in diesem Modell keine Möglichkeit, um von (\textbf{d}) auf
(\textbf{c}) oder (\textbf{b}) zu kommen.

\subsection{Availability}\label{sec:sol_avail}
\paragraph{Einfache Variante:}
\begin{figure}
    \centering
    \begin{subfigure}[b]{0.48\linewidth}
        \centering
        \begin{tikzpicture}[>=latex]
            \tikzstyle{markov} = [top color=blue!30, bottom color=white, 
                            draw=blue!50!black!100, drop shadow]
            \tikzstyle{bad} = [top color=red!30]

            \node[markov,circle split,label=below left:{\color{green!70!black}\textbf{a}}] (1) {$3/3$ \nodepart{lower} $1/1$};
            \node[markov,bad,circle split,label=below right:{\color{green!70!black}\textbf{b}}] (2) [above right=of 1]{$2/3$ \nodepart{lower} $1/1$};
            \node[markov,bad,circle split,label=below right:{\color{green!70!black}\textbf{c}}] (3) [below right=of 1]{$3/3$ \nodepart{lower} $0/1$};
            \draw[->] (1) to node[sloped,above]{$3\lambda_R$}(2);
            \draw[->] (1) to node[sloped,above]{$\lambda_N$}(3);
            \draw[->, blue!70!black] (3) to[bend right=45] node[sloped,above]{$\mu_N$}(1);
            \draw[->, blue!70!black] (2) to[bend right=45] node[sloped,above]{$\mu_R$}(1);
        \end{tikzpicture}
        \caption{\textbf{Zustand a}: Funktionierend, \\
            \textbf{Zustand b}: Rechnerausfall, \\
            \textbf{Zustand c}: Netzwerkausfall}
        \label{fig:states_simple_avail}
    \end{subfigure}
    \begin{subfigure}[b]{0.48\linewidth}
        \centering
            \begin{tikzpicture}[>=latex]
                \tikzstyle{markov} = [top color=blue!30, bottom color=white, 
                                draw=blue!50!black!100, drop shadow]
                \tikzstyle{bad} = [top color=red!30]

                \node[markov,circle,label=below right:{\color{green!70!black}\textbf{a}}] (1) {alive};
                \node[markov,bad,circle,label=below right:{\color{green!70!black}\textbf{b}}] (2)
                [right=5 of 1]{dead};
                \draw[->] (1) to node[sloped,above]{$3\lambda_R + \lambda_N$}(2);
                \draw[->, blue!70!black] (2) to[bend right=45] node[sloped,above]{$\mu_R=\mu_N$}(1);
            \end{tikzpicture}
            \caption{Vereinfachte Variante von \Fref{fig:states_simple_avail} \\
                \textbf{Zustand a}: Funktionierend, \\
                \textbf{Zustand b}: nicht Funktionierend}
            \label{fig:states_simple_avail_simply}
    \end{subfigure}
    \caption{Zustände der einfachen Variante zur Berechnung der Availability.\\
        Die Zustände in denen das System lauffähig ist, sind hier Blau
    dargestellt.}
    \label{fig:states_simple_avail_comb}
\end{figure}
Zur Berechnung der Availability besitzt dieses Modell die selben Zustände, die
auch zur Berechnung der MTTF verwendet wurden. Jedoch wird zusätzlich ein
Übergang vom Zustand \enquote{dead} zu den Zuständen \enquote{alive} eingeführt.
Auch hier wird angenommen, dass der Techniker alle Fehler gleichzeitig beheben
kann und somit direkt in den Zustand \enquote{alive}(alle Komponenten
funktionieren) wechseln kann.
(siehe \Fref{fig:states_simple_avail_comb}).

\paragraph{Redundante Variante:}
\begin{figure}
    \centering
        \begin{tikzpicture}[>=latex]
            \tikzstyle{markov} = [top color=blue!30, bottom color=white, 
                            draw=blue!50!black!100, drop shadow]
            \tikzstyle{bad} = [top color=red!30]

            \node[markov,circle split,label=below left:{\color{green!70!black}\textbf{a}}] (42) {$4/4$ \nodepart{lower} $2/2$};
            \node[markov,circle split,label=above right:{\color{green!70!black}\textbf{b}}] (32) [above right=of 42]{$3/4$ \nodepart{lower} $2/2$};
            \node[markov,circle split,label=below right:{\color{green!70!black}\textbf{d}}] (31) [below right=of 32]{$3/4$ \nodepart{lower} $1/2$};
            \node[markov,circle split,label=below right:{\color{green!70!black}\textbf{c}}] (41) [below right=of 42]{$4/4$ \nodepart{lower} $1/2$};
            \node[markov,bad,circle,label=below right:{\color{green!70!black}\textbf{e}}] (d)  [right=3 of 31]{dead};
            \draw[->] (42) to node[sloped,above]{$4\lambda_R$}(32);
            \draw[->] (32) to[bend right=45] node[sloped,above]{$\mu_R$}(42);
            \draw[->] (42) to node[sloped,above]{$2\lambda_N$}(41);
            \draw[->] (41) to[bend right=-45] node[sloped,above]{$\mu_N$}(42);
            \draw[->] (41) to node[sloped,above]{$4\lambda_R$}(31);
            \draw[->] (31) to node[sloped,above]{$\mu_N$} (42);
            \draw[->] (32) to node[sloped,above]{$2\lambda_N$}(31);
            \draw[->] (32) to[bend left=20] node[sloped,above]{$3\lambda_R$}(d);
            \draw[->] (31) to node[sloped,above]{$3\lambda_R+\lambda_N$}(d);
            \draw[->] (41) to[bend right=20] node[sloped,above]{$\lambda_N$}(d);
            \draw[->,blue!70!black] (d) to[bend right=108] node[sloped,above]{$\mu_N=\mu_R$}(42);
        \end{tikzpicture}
    \caption{Zustände der redundanten Variante zur Berechnung der Availability.\\
        Die Zustände in denen das System lauffähig ist, sind hier Blau
    dargestellt.}
    \label{fig:states_redundant_avail}
\end{figure}
Zur Berechnung der Availability von diesem System wird hier ebenso ein Übergang von \enquote{dead} 
zum Startzustand eingeführt. 
Für die Berechnung der Verfügbarkeit wird definiert, dass das System noch
lauffähig ist, wenn es sich in einem der blauen Zustände befindet. (siehe
\Fref{fig:states_redundant_avail}).

\paragraph{Kostenvergleich:}
Um nun die Kosten der beiden Varianten zu vergleichen, muss von beiden Systemen
die Availability berechnet werden. (\Fref{sec:res_avail}) Erst dann können die Ausfallskosten zu den
laufenden Kosten in Relation gestellt und verglichen werden. (siehe
\Fref{sec:cost_calc})

\section{Ergebnisse}

\subsection{MTTF}\label{sec:results_mttf}
\begin{figure}
\centering
\begin{tikzpicture}
    \begin{axis}[
        height=5.2cm, width=12cm,
        ymin=0, ymax=1,
        axis lines=center,
        axis on top=true,
        domain=0:1,
        legend pos=south east,
        %legend style={at={(axis cs:1800,0.2)},anchor=south east},
        xlabel=Tage,
        ylabel=Ausfallswahrscheinlichkeit,
        x label style={at={(axis description cs:1.07,0.07)},anchor=north},
        y label style={at={(axis description cs:-0.07,.5)},rotate=90,anchor=south},
        %ylabel=Number of failed banks
        ] % use TeX as calculator: 
    \addplot[blue,thick,domain=0:1800,samples=100] table {simple_mttf.dat}; 
    \addlegendentry{Simple Variante}
    \addplot[green,thick,domain=0:1800,samples=100] table {redundant_without_repair_mttf.dat}; 
    \addlegendentry{Red. Variante (o. Rep.)}
    \addplot[red,thick,domain=0:1800,samples=100] table {redundant_mttf.dat}; 
    \addlegendentry{Red. Variante (m. Rep.)}
    \end{axis} 
\end{tikzpicture}
\caption{Vergleich der MTTF der simplen Variante (siehe
    \Fref{fig:states_simple_mttf_simply}), \\
    der redundanten Variante ohne Reparatur (siehe
    \Fref{fig:states_redundant_without_rep})\\
    und der redundanten Variante mit Reparatur (siehe \Fref{fig:states_redundant_with_rep})}
\label{fig:mttf_result}
\end{figure}
TODO: viel BLAHBLAH

Die Ergebnisse von MTTF sind in \Fref{fig:mttf_result} dargestellt.
Diese Ergebnisse wurden durch \Fref{sec:calc_mttf_simple},
\Fref{sec:calc_mttf_redundant} und \Fref{sec:calc_mttf_redundant_rep} berechnet.

\subsection{Availability}\label{sec:res_avail}
TODO: viel blah blah
Die Availability $\alpha_s$ des einfachen Systems lässt sich als Zustandswahrscheinlichkeit der Markov Kette (siehe \Fref{fig:states_simple_avail_simply}) des Zustands \textbf{b} berechnen. Dies kann mit dem \emph{SHARPE} Befehl \mbox{\texttt{expr prob(SYS,1)}} bewerkstelligt werden. \\
Mit den gegebenen Werte ergibt sich: $\alpha_s = \printpercent{\alphas}$\\

Die Availability $\alpha_r$ des fehlertoleranten Systems entspricht der Wahrscheinlichkeit, sich in einem der Zustände $\lbrace a, b, c, d \rbrace$ zu befinden (siehe \Fref{fig:states_redundant_avail}). Dies ist äquivalent mit Zustandswahrscheinlichkeit nicht im Zustand \textbf{e} (System Ausfall) zu sein und kann mittels \mbox{\texttt{expr 1-prob(SYS,5)}} berechnet werden (Zustand 5 = Zustand e).\\
In unserem Fall ergibt das: $\alpha_r = \printpercent{\alphar}$\\
\section{Diskussion}
TODO:
\subsection{Kostenrechnung}\label{sec:cost_calc}
Aus der Simulation mit \emph{SHARPE} haben wir die beiden Availability Werte $\alpha_s$ (für das einfache System) und $\alpha_r$ für das Redundante System erhalten.\\
Sei $\Omega$ die Betriebskosten des einfachen Systems, so lassen sich die Kosten beschreiben mit:
\begin{equation}
C_s(k) = \Omega \cdot \alpha_s + (1-\alpha_s)\cdot \Omega \cdot k
\end{equation}

wobei $k \cdot \Omega$ die Ausfallkosten des Systems sind ($k \in
\mathbb{N}^+$).\\$k$ stellt das Verhältnis zwischen Laufender Kosten und Ausfallkosten des einfachen Systems dar.\\
Die Kosten des Redundanten Systems ergeben sich mit: 
\begin{equation}
C_r(k) = \Omega \cdot \printCostFactor \alpha_r + (1- \alpha_r) \cdot \Omega \cdot k
\end{equation}

Um das Verhältnis $k$ zu bestimmen, ab dem sich der Betrieb des Redundanten Systems auszahlt, muss der Schnittpunkt $k_x$ bestimmt werden:
\begin{equation}
\begin{split}
C_s(k_x) = C_r(k_x) \\ &
 \Leftrightarrow
\Omega \cdot \alpha_s + (1-\alpha_s)\Omega \cdot k_x = \Omega \cdot \printCostFactor \alpha_r + (1- \alpha_r) \cdot \Omega \cdot k_x \\ & \Leftrightarrow
k_x \cdot (1 - \alpha_s - 1 + \alpha_r) = \printCostFactor \alpha_r - \alpha_s \\ & \Leftrightarrow
k_x = \frac{\printCostFactor \alpha_r - \alpha_s}{\alpha_r - \alpha_s}
\end{split}
\end{equation}

\begin{figure}
\centering
\begin{tikzpicture}
    \begin{axis}[
	    height=5.2cm, width=12cm,
		axis lines=center,
		axis on top=true,
		domain=0:100,
    	xmin=0, xmax=100,
    	xlabel=k,
        ymin=0, ymax=50,
        ylabel=Kosten $\Omega$,
        ytick = {0,10,...,50},
        x label style={at={(axis description cs:1.05,0.05)},anchor=north},
        y label style={at={(axis description cs:-0.07,.5)},rotate=90,anchor=south},
        legend style={at={(axis cs:100, 2)},anchor=south east},
        %ylabel=Number of failed banks
        ] % use TeX as calculator: 
    \addplot +[mark=none] {kst1(\workingCosts, \alphas, x)};
    \addlegendentry{Kosten simple}
    \addplot +[mark=none] {kst2(\workingCosts, \alphar, x)};
    \addlegendentry{Kosten redundant}
    \draw[green] (axis cs:\punktkx,0) -- (axis cs:\punktkx,35) node [above]{$k_x$};
    \end{axis} 
\end{tikzpicture}
\caption{Vergleich der Kostenfunktionen mit Schnittpunkt $k_x$}
\label{fig:cost_result}
\end{figure}


%Enable fixpoint-arithmetic only for this short calculation:
\pgfkeys{/pgf/fpu=true}
\pgfkeys{/pgf/number format,use comma,fixed,precision=5}

Mit den beiden von \emph{SHARPE} berechneten Werten ergibt sich:
\[k_x = \frac{\printCostFactor \cdot \printpercent{\alphar} -
\printpercent{\alphas}}{\printpercent{\alphar} - \printpercent{\alphas}} =
\pgfmathprintnumber[fixed, precision=3]{\punktkx} \approx
\pgfmathprintnumber[fixed, precision=0]{\punktkx} \]

\pgfkeys{/pgf/fpu=false}

\section{Fazit}
TODO: blahblah
\newpage
\appendix
\section{Sharpe-Quellcode}
\subsection{MTTF}
\subsubsection{MTTF Ohne Redundanz}\label{sec:calc_mttf_simple}
\lstinputlisting{simple_mttf.shp}

\subsubsection{MTTF mit Redundanz inklusive Reparatur}\label{sec:calc_mttf_redundant_rep}
\lstinputlisting{redundant_mttf.shp}

\subsubsection{MTTF mit Redundanz ohne Reparatur}\label{sec:calc_mttf_redundant}
\lstinputlisting{redundant_without_repair_mttf.shp}
\newpage
\subsection{Availability}
\subsubsection{Availability Ohne Redundanz}
\lstinputlisting{simple_availability.shp}

\subsubsection{Availability mit Redundanz}
\lstinputlisting{redundant_availability.shp}
\end{document}
